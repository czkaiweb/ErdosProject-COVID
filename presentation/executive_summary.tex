\documentclass{article}

%\usepackage[margin=1.2in]{geometry}

\usepackage{hyperref,color}
\usepackage[usenames,dvipsnames]{xcolor}
 \hypersetup{pdftitle={},pdfauthor={},colorlinks=true,linkcolor=MidnightBlue,citecolor=MidnightBlue}

\begin{document}
\title{Executive Summary:\\Team Octopus COVID-19 Hospitalization Forecaster}
%\author{Julian Rosen}
\date{}
\maketitle
%\begin{abstract}
%\end{abstract}
\thispagestyle{empty}
The COVID-19 pandemic puts unprecedented strain on the American healthcare system, and the response is harder because the amount of stress caused by COVID-19 changes quickly. State and local health authorities need to issue guidances on elective procedures and staffing, based on guesses.
\bigskip

\noindent There are many models out there predicting COVID-19 case rates. However, predicting cases has several drawbacks that make it less useful to the healthcare industry.
\begin{itemize}
\item The time from diagnosis to hospitalization can vary from no time to several weeks.
\item The percentage of cases that result in hospitalization changes over time, and is know to depend substantially on vaccination status, people's behavior, etc.
\end{itemize}
\bigskip

Enter: The Team Octopus Covid Hospitalization Forecaster! We trained a machine learning model to predict the number of hospital beds needed for COVID-19 patients, up to 14 days in the future.
\smallskip

\noindent Benefits of our forecaster:
\begin{itemize}
\item It cuts out the noise in the data caused by mild cases and just highlights the needs of patients with severe COVID-19.
\item The predictions are actionable: they can help healthcare providers figure out when to request or donate resources.
\item We know our data is high quality because it comes directly from the CDC.
\item The software is free and open source\\
\url{https://github.com/czkaiweb/ErdosProject-COVID/}
\item The tool is easy to use (visit our website \url{http://130.111.196.10/}).
\end{itemize}


%\bibliographystyle{hplain}
%\bibliography{jrbiblio}
\end{document}